\documentclass{article}
\usepackage[french]{babel}
\usepackage[utf8]{inputenc}
\usepackage[T1]{fontenc}
 \usepackage{amsmath}
 \usepackage{amsfonts}
 \usepackage{amssymb}
\usepackage[active,tightpage]{preview}

 \newcommand{\bigO}[1]{\ensuremath{\mathop{}\mathopen{}O\mathopen{}\left(#1\right)}}
 \newcommand{\reals} {\rm I\!R}
  \newcommand{\nats} {\rm I\!N}

\usepackage[french, linesnumbered, lined, algoruled, titlenotnumbered,shortend]{algorithm2e}
\begin{document}
\begin{preview}
Notation $\Omega$: "grand omega"
    $$\Omega(f(n)) := \{g: \mathbb{N} \rightarrow \mathbb{R}^* | \exists c \in \mathbb{R}^*, n_0 \in \mathbb{N} \backslash \forall n \geq n_0, g(n) \geq c~f(n)\}$$
    $Ops(n) \in \Omega(n)$ dire "Ops(n) est en grand omega de n"\\
    $\Rightarrow$ Borne inférieure asymptotique sur nombre d'opérations (au coef. mult. $c$ près)\\
    %\vspace{3mm}
    Ex:  $Ops(n) = 10 n^2 - 5 n + 2$
    \begin{minipage}{0.5\textwidth}
    \begin{itemize}
        \item $Ops(n) \in \Omega(n^2)$
        \item $Ops(n) \in \Omega(n)$
        \item $Ops(n) \not\in \Omega(n^3)$
    \end{itemize}
    \end{minipage}
 \end{preview}
\end{document}